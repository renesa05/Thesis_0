%%-------------Abstract-----------------
\doublespacing
\setlength{\topmargin}{-.5in}
\chapter*{Abstract}


\addcontentsline{toc}{chapter}{Abstract}


%enter text for the abstract below

A natural question when dealing with problem that are NP hard to solve is whether it's possible to compute a solution which is close enough to the optimal solution for practical purposes. The usual notion of "closeness" is that the value of the solution is not far from the optimal value. In this M.Sc thesis, we will focus on the generalization of this notion where closeness is defined with respect to a given distance function. This framework, named "Structure approximation", was introduced in 2007 paper by Hamilton, M{\"u}ller , van Rooij and Wareham, who posed the question of complexity of solving NP-hard optimization problems in this setting.


\
 
Here, we are going to examine the complexity of structure approximation and show that at least for some choices for distance function any non trivial structure approximation is as hard to compute as the original problem. In particular we will show the most natural distance function in a computational complexity setting treated as binary strings which is Hamming distance. We will survey the relevant results from other related problems and show how they relate to our problem. We will also survey the techniques that can be used to solve this problem to prove inapproximability results such as Error correcting codes. 


